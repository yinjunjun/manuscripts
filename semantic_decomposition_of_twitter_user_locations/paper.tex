\documentclass[a4paper,11pt]{article}
\usepackage[margin=1in]{geometry}
\usepackage{lastpage,fancyhdr,graphicx}
\graphicspath{ {/figure/} }
\usepackage{authblk}

\begin{document}
% \title{A data synthesis approach for semantic decompositions of Twitter user locations: A case study of geo-located Twitter data in Chicago}
\title{A tale of two cities: A GIS based data synthesis approach to understanding semantic decompositions of Twitter user locations}
\author{Junjun Yin}
\author{Shaowen Wang~\thanks{shaowen@illinois.edu}}
\affil[1]{Department of Geography and Geographic Information Science}
\affil[ ]{University of Illinois at Urbana-Champaign, IL, 61801, USA}
\affil[2]{CyberGIS Center for Advanced Digital and Spatial Studies}
\affil[3]{Department of Geography and Geographic Information Science}
\affil[4]{National Center for Supercomputing Applications}
\affil[ ]{University of Illinois at Urbana-Champaign, IL, 61801, USA}
\renewcommand\Authands{ and }
\maketitle

\section*{Abstract}
Semantic Trajectory Mining for Location Prediction

Today's pervasive Location Based Social Media provide abundant user-generated geographic information
In this paper, we present a geographical information system (GIS) based data synthesis approch to understanding semantic decompositions of Twitter user locations.
Specifically, we implemented a
This approach 
We analyzed the 

\section{Introduction}
With the advancements in mobile technology, the location information 
what is it?

why it is important?

why is my solution?

The remainder of this paper is organized as follows. Section 2 introduces the related work 
Section 3 details the 
Detailed analysis of the
Section 5 presents the results
Section 6 concludes the paper.

\section{Related Work}



\section{Methods and Materials}
\subsection{Geo-located Twitter Data}
Geo-located Twitter data refer to  


\subsection{A scalable data synthesis framework}



\section{Results}


\section{Discussions and Conclusions}
Talk about OpenStreetMap for potential large scale validation
talk about the limitation of detailed land use maps currently available.



\bibliographystyle{apalike}
\bibliography{ref}

\end{document}