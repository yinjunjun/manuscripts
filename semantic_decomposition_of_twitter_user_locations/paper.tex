\documentclass[a4paper, 11pt]{article}
\usepackage{authblk}
\usepackage[round, sort, numbers, authoryear]{natbib}
\usepackage{regexpatch}
\usepackage{graphicx}
\graphicspath{ {/figures/} }
\usepackage[section]{placeins}
\usepackage{setspace}
\usepackage[margin=1in]{geometry}
\usepackage[]{algorithm2e}
\usepackage{caption}
\usepackage{subcaption}
\date{}

\usepackage{lineno}

\begin{document}
% \title{A data synthesis approach for semantic decompositions of Twitter user locations: A case study of geo-located Twitter data in Chicago}
\title{A tale of two cities: A GIS based data synthesis approach to understanding semantic decompositions of Twitter user locations}
\author[1~\thanks{Corresponding author: jyin@psu.edu}]{Junjun Yin}
\affil[ ]{Social Science Research Institute}
\affil[ ]{The Pennsylvania State University, State College, 16801, PA, USA}
\renewcommand\Authands{ and }
\maketitle

\section*{Abstract}
Human mobility is as complex human behavior, by its nature difficulty to study, confounded by human variability.

With the advancement positioning technology, the


Today's pervasive Location Based Social Media provide abundant user-generated geographic information
In this paper, we present a geographical information system (GIS) based data synthesis approach to understanding semantic decompositions of Twitter user locations.
Specifically, we implemented a
This approach 
We analyzed the 



\section{Introduction}

xx~\citep{yin2016exploring}

With the advancements in mobile technology, the location information 
what is it?

why it is important?

why is my solution?

The remainder of this paper is organized as follows. Section 2 introduces the related work 
Section 3 details the 
Detailed analysis of the
Section 5 presents the results
Section 6 concludes the paper.

\section{Related Work}
yy~\cite{yin2017depicting}


\section{Data and Methods}



\subsection{Geo-located Twitter Data}
Geo-located Twitter data refer to  


\subsection{A scalable data synthesis framework}


\section{Results}


\section{Discussions and Conclusions}
Talk about OpenStreetMap for potential large scale validation
talk about the limitation of detailed land use maps currently available.



\bibliographystyle{apalike}
\bibliography{ref}
\end{document}