% Template article for preprint document class `elsart'
% SP 2006/04/26

\documentclass{elsart}

% Use the option doublespacing or reviewcopy to obtain double line spacing
% \documentclass[doublespacing]{elsart}

% if you use PostScript figures in your article
% use the graphics package for simple commands
% \usepackage{graphics}
% or use the graphicx package for more complicated commands
% \usepackage{graphicx}
% or use the epsfig package if you prefer to use the old commands
% \usepackage{epsfig}

% The amssymb package provides various useful mathematical symbols
\usepackage{amssymb}

% The lineno packages adds line numbers. Start line numbering with
% \begin{linenumbers}, end it with \end{linenumbers}. Or switch it on
% for the whole article with \linenumbers.
% \usepackage{lineno}

% \linenumbers
\begin{document}

\begin{frontmatter}

% Title, authors and addresses

% use the thanksref command within \title, \author or \address for footnotes,
% use the corauthref command within \author for corresponding author footnotes,
% use the ead command for the email address,
% and the form \ead[url] for the home page:
% \title{Title\thanksref{label1}}
% \thanks[label1]{}
% \author{Name\corauthref{cor1}\thanksref{label2}}
% \ead{email address}
% \ead[url]{home page}
% \thanks[label2]{}
% \corauth[cor1]{}
% \address{Address\thanksref{label3}}
% \thanks[label3]{}

\title{Exploring Multi-Scale Spatiotemporal Twitter User Mobility Patterns}

% use optional labels to link authors explicitly to addresses:
% \author[label1,label2]{}
% \address[label1]{}
% \address[label2]{}

\author{}

\address{}

\begin{abstract}
% Text of abstract
\end{abstract}

\begin{keyword}
% keywords here, in the form: keyword \sep keyword

% PACS codes here, in the form: \PACS code \sep code
\PACS 
\end{keyword}
\end{frontmatter}

% main text
\section{}
\label{}

% The Appendices part is started with the command \appendix,
% appendix sections are then done as normal sections
% \appendix

% \section{}
% \label{}

\begin{thebibliography}{00}
\bibitem{zheng2008understanding}
Zheng, Y., Li, Q., Chen, Y., Xie, X., Ma, W.Y., 2008. Understanding mobility based on GPS data. Proceedings of the 10th international conference on Ubiquitous computing. ACM, pp. 312--321.

\bibitem{jiang2009characterizing}
Jiang, B., Yin, J., Zhao, S., 2009. Characterizing the human mobility pattern in a large street network. {\em Physical Review E}, {\em 80},~021136.

\bibitem{belik2011natural}
Belik, V., Geisel, T., Brockmann, D., 2011. Natural human mobility patterns and spatial spread of infectious diseases. {\em Physical Review X}, {\em 1},~011001.

\bibitem{greenwood1985human}
Greenwood, M.J., 1985. Human migration: Theory, models, and empirical studies. {\em Journal of regional Science}, {\em 25}, pp. 521--544.

\bibitem{brockmann2006scaling}
Brockmann, D., Hufnagel, L., Geisel, T., 2006. The scaling laws of human travel. {\em Nature}, {\em 439}, pp. 462--465.

\bibitem{gonzalez2008understanding}
Gonzalez, M.C., Hidalgo, C.A., Barabasi, A.L., 2008. Understanding individual human mobility patterns. {\em Nature}, {\em 453}, pp. 779--782.

\bibitem{Jurdak2015}
Jurdak, R., Zhao, K., Liu, J., AbouJaoude, M., Cameron, M., Newth, D., 2015. Understanding Human Mobility from Twitter. {\em PLoS ONE}, {\em 10},~e0131469.

\bibitem{rhee2011levy}
Rhee, I., Shin, M., Hong, S., Lee, K., Kim, S.J., Chong, S., 2011. On the levy-walk nature of human mobility. {\em IEEE/ACM transactions on networking (TON)}, {\em 19}, pp. 630--643.

\bibitem{sevtsuk2010does}
Sevtsuk, A., Ratti, C., 2010. Does urban mobility have a daily routine? Learning from the aggregate data of mobile networks. {\em Journal of Urban Technology}, {\em 17}, pp. 41--60.

\bibitem{kung2014exploring}
Kung, K.S., Greco, K., Sobolevsky, S., Ratti, C. 2014. Exploring universal patterns in human home-work commuting from mobile phone data. {\em PLoS ONE}, {\em 9},~e96180.

\bibitem{thatcher2014living}
Thatcher, J., 2014. Living on fumes: Digital footprints, data fumes, and the limitations of spatial big data. {\em International Journal of Communication}, {\em 8}, pp. 1765--1783.

\bibitem{hawelka2014geo}
Hawelka, B., Sitko, I., Beinat, E., Sobolevsky, S., Kazakopoulos, P., Ratti, C., 2014. Geo-located Twitter as proxy for global mobility patterns.{\em Cartography and Geographic Information Science}, {\em
  41}, pp. 260--271.

\bibitem{giannotti2008mobility}
Giannotti, F., Pedreschi, D., 2008.{\em Mobility, data mining and privacy: Geographic knowledge discovery}, Springer Science \& Business Media.

\bibitem{crampton2014collect}
Crampton, J.W. 2014. Collect it all: national security, Big Data and governance. {\em GeoJournal}, pp. 1--13.

\bibitem{wu2014intra}
Wu, L., Zhi, Y., Sui, Z., Liu, Y., 2014. Intra-urban human mobility and activity transition: Evidence from social media check-in data. {\em PLoS ONE}, {\em 9},~e97010.

\bibitem{hasan2013understanding}
Hasan, S., Zhan, X., Ukkusuri, S.V., 2013. Understanding urban human activity and mobility patterns using large-scale location-based data from online social media. Proceedings of the 2nd ACM SIGKDD international workshop on urban computing. ACM,  2013, p.~6.

\bibitem{cho2011friendship}
Cho, E., Myers, S.A., Leskovec, J. 2011, Friendship and mobility: user movement in location-based social networks. Proceedings of the 17th ACM SIGKDD international conference on Knowledge discovery and data mining. ACM,  2011, pp. 1082--1090.

\bibitem{noulas2012tale}
Noulas, A., Scellato, S., Lambiotte, R., Pontil, M., Mascolo, C., 2012. A tale of many cities: universal patterns in human urban mobility. {\em PLoS ONE}, {\em 7},~e37027.

\bibitem{balcan2009multiscale}
Balcan, D., Colizza, V., Gon{\c{c}}alves, B., Hu, H., Ramasco, J.J.,
  Vespignani, A., 2009. Multiscale mobility networks and the spatial spreading of infectious diseases.{\em Proceedings of the National Academy of Sciences},{\em 106}, pp. 21484--21489.

\bibitem{tamerius2011global}
Tamerius, J., Nelson, M.I., Zhou, S.Z., Viboud, C., Miller, M.A., Alonso, W.J., 2011. Global influenza seasonality: reconciling patterns across temperate and tropical regions.{\em Environmental health perspectives}, {\em 119},~439.

\bibitem{tsou2015}
Tsou, M.H., 2015. Research challenges and opportunities in mapping social media and Big Data.{\em Cartography and Geographic Information Science} {\bf 2015}, {\em 42}, pp. 70--74.

\bibitem{zheng2010geolife}
Zheng, Y., Xie, X., Ma, W.Y., 2010. GeoLife: A Collaborative Social Networking Service among User, Location and Trajectory.

\bibitem{becker2013human}
Becker, R., C{\'a}ceres, R., Hanson, K., Isaacman, S., Loh, J.M., Martonosi,
  M., Rowland, J., Urbanek, S., Varshavsky, A., Volinsky, C., 2013. Human mobility characterization from cellular network data.{\em Communications of the ACM}, {\em 56}, pp. 74--82.

\bibitem{sobolevsky2013delineating}
Sobolevsky, S., Szell, M., Campari, R., Couronn{\'e}, T., Smoreda, Z., Ratti,
  C., 2013. Delineating geographical regions with networks of human interactions in an extensive set of countries.{\em PLoS ONE}, {\em 8},~e81707.

\bibitem{twitterAPI}
Twitter, 2016. Twitter streaming API.{\em Available from: https://dev.twitter.com/streaming/overview}.

\bibitem{cranshaw2012livehoods}
Cranshaw, J., Schwartz, R., Hong, J.I., Sadeh, N.M., 2012. The Livehoods Project: Utilizing Social Media to Understand the Dynamics of a City. ICWSM.

\bibitem{mitchell2013geography}
Mitchell, L., Frank, M.R., Harris, K.D., Dodds, P.S., Danforth, C.M., 2013. The geography of happiness: Connecting twitter sentiment and expression, demographics, and objective characteristics of place.

\bibitem{longley2015geotemporal}
Longley, P.A., Adnan, M., Lansley, G., others., 2015. The geotemporal demographics of Twitter usage. {\em Environment and Planning A}, {\em 47}, pp. 465--484.

\bibitem{hagerstrand1985time}
H{\"a}gerstrand, T., others., 1985. Time-geography: focus on the corporeality of man, society, and environment. {\em The science and praxis of complexity}, pp. 193--216.

\bibitem{kwan2004geovisualization}
Kwan, M.P., Lee, J., 2004. Geovisualization of human activity patterns using 3D GIS: a time-geographic approach. {\em Spatially integrated social science}, {\em 27}.

\bibitem{andrienko2007designing}
Andrienko, N., Andrienko, G., 2007. Designing visual analytics methods for massive collections of movement data. {\em Cartographica: The International Journal for Geographic Information and Geovisualization}, {\em 42}, pp. 117--138.

\bibitem{maceachren2001research}
MacEachren, A.M., Kraak, M.J., 2001. Research challenges in geovisualization. {\em Cartography and Geographic Information Science}, {\em
  28}, pp. 3--12.

\bibitem{maceachren2004maps}
MacEachren, A.M., 2004. {\em How maps work: representation, visualization, and design}, Guilford Press.

\bibitem{andrienko2007visual}
Andrienko, G., Andrienko, N., Wrobel, S., 2007. Visual analytics tools for analysis of movement data. {\em ACM SIGKDD Explorations Newsletter}, {\em 9}, pp. 38--46.

\bibitem{black2012twitter}
Black, A., Mascaro, C., Gallagher, M., Goggins, S.P., 2012. Twitter zombie: Architecture for capturing, socially transforming and analyzing the Twittersphere. Proceedings of the 17th ACM international conference on Supporting group work. ACM, pp. 229--238.

\bibitem{shvachko2010hadoop}
Shvachko, K., Kuang, H., Radia, S., Chansler, R., 2010. The hadoop distributed file system. Mass Storage Systems and Technologies (MSST), 2010 IEEE 26th Symposium on. IEEE, 2010, pp. 1--10.

\bibitem{dean2008mapreduce}
Dean, J., Ghemawat, S., 2008. MapReduce: simplified data processing on large clusters. {\em Communications of the ACM}, {\em 51}, pp. 107--113.

\bibitem{gao2012exploring}
Gao, H., Tang, J., Liu, H., 2012. Exploring Social-Historical Ties on Location-Based Social Networks. ICWSM.

\bibitem{buttenfield1991map}
Buttenfield, B.P., McMaster, R.B., 1991. {\em Map Generalization: Making rules for knowledge representation}, Longman Scientific \& Technical New York.

\bibitem{samet1984quadtree}
Samet, H., 1984. The quadtree and related hierarchical data structures. {\em ACM Computing Surveys (CSUR)}, {\em 16}, pp. 187--260.

\bibitem{clauset2009power}
Clauset, A., Shalizi, C.R., Newman, M.E., 2009. Power-law distributions in empirical data. {\em SIAM review}, {\em 51}, pp. 661--703.

\bibitem{reynolds2012truncated}
Reynolds, A., 2012. Truncated L{\'e}vy walks are expected beyond the scale of data collection when correlated random walks embody observed movement patterns. {\em Journal of The Royal Society Interface}, {\em
  9}, pp. 528--534.

\bibitem{zhao2015explaining}
Zhao, K., Musolesi, M., Hui, P., Rao, W., Tarkoma, S., 2015. Explaining the power-law distribution of human mobility through transportation modality decomposition. {\em Scientific reports}, {\em 5}.



\end{thebibliography}

\end{document}

